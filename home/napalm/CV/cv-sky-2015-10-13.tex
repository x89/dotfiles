%%%%%%%%%%%%%%%%%%%%%%%%%%%%%%%%%%%%%%%%%%%%%%%%%%%%%%%%%%%%%%%%%%%%%%%%
%%%%%%%%%%%%%%%%%%%%%% Simple LaTeX CV Template %%%%%%%%%%%%%%%%%%%%%%%%
%%%%%%%%%%%%%%%%%%%%%%%%%%%%%%%%%%%%%%%%%%%%%%%%%%%%%%%%%%%%%%%%%%%%%%%%

%%%%%%%%%%%%%%%%%%%%%%%%%%%%%%%%%%%%%%%%%%%%%%%%%%%%%%%%%%%%%%%%%%%%%%%%
%% NOTE: If you find that it says                                     %%
%%                                                                    %%
%%                           1 of ??                                  %%
%%                                                                    %%
%% at the bottom of your first page, this means that the AUX file     %%
%% was not available when you ran LaTeX on this source. Simply RERUN  %%
%% LaTeX to get the ``??'' replaced with the number of the last page  %%
%% of the document. The AUX file will be generated on the first run   %%
%% of LaTeX and used on the second run to fill in all of the          %%
%% references.                                                        %%
%%%%%%%%%%%%%%%%%%%%%%%%%%%%%%%%%%%%%%%%%%%%%%%%%%%%%%%%%%%%%%%%%%%%%%%%

%%%%%%%%%%%%%%%%%%%%%%%%%%%% Document Setup %%%%%%%%%%%%%%%%%%%%%%%%%%%%

% Don't like 10pt? Try 11pt or 12pt
\documentclass[12pt]{article}

% This is a helpful package that puts math inside length specifications
\usepackage{calc}
\usepackage[utf8]{inputenc}

% Simpler bibsection for CV sections
% (thanks to natbib for inspiration)
\makeatletter
\newlength{\bibhang}
\setlength{\bibhang}{1em}
\newlength{\bibsep}
 {\@listi \global\bibsep\itemsep \global\advance\bibsep by\parsep}
\newenvironment{bibsection}%
        {\vspace{-\baselineskip}\begin{list}{}{%
       \setlength{\leftmargin}{\bibhang}%
       \setlength{\itemindent}{-\leftmargin}%
       \setlength{\itemsep}{\bibsep}%
       \setlength{\parsep}{\z@}%
        \setlength{\partopsep}{0pt}%
        \setlength{\topsep}{0pt}}}
        {\end{list}\vspace{-.6\baselineskip}}
\makeatother

% Layout: Puts the section titles on left side of page
\reversemarginpar

%
%         PAPER SIZE, PAGE NUMBER, AND DOCUMENT LAYOUT NOTES:
%
% The next \usepackage line changes the layout for CV style section
% headings as marginal notes. It also sets up the paper size as either
% letter or A4. By default, letter was used. If A4 paper is desired,
% comment out the letterpaper lines and uncomment the a4paper lines.
%
% As you can see, the margin widths and section title widths can be
% easily adjusted.
%
% ALSO: Notice that the includefoot option can be commented OUT in order
% to put the PAGE NUMBER *IN* the bottom margin. This will make the
% effective text area larger.
%
% IF YOU WISH TO REMOVE THE ``of LASTPAGE'' next to each page number,
% see the note about the +LP and -LP lines below. Comment out the +LP
% and uncomment the -LP.
%
% IF YOU WISH TO REMOVE PAGE NUMBERS, be sure that the includefoot line
% is uncommented and ALSO uncomment the \pagestyle{empty} a few lines
% below.
%

%% Use these lines for letter-sized paper
\usepackage[paper=letterpaper,
            %includefoot, % Uncomment to put page number above margin
            marginparwidth=1.2in,     % Length of section titles
            marginparsep=.05in,       % Space between titles and text
            margin=1in,               % 1 inch margins
            includemp]{geometry}

%% Use these lines for A4-sized paper
%\usepackage[paper=a4paper,
%            %includefoot, % Uncomment to put page number above margin
%            marginparwidth=30.5mm,    % Length of section titles
%            marginparsep=1.5mm,       % Space between titles and text
%            margin=25mm,              % 25mm margins
%            includemp]{geometry}

%% More layout: Get rid of indenting throughout entire document
\setlength{\parindent}{0in}

%% This gives us fun enumeration environments. compactitem will be nice.
\usepackage{paralist}

%% Reference the last page in the page number
%
% NOTE: comment the +LP line and uncomment the -LP line to have page
%       numbers without the ``of ##'' last page reference)
%
% NOTE: uncomment the \pagestyle{empty} line to get rid of all page
%       numbers (make sure includefoot is commented out above)
%
\usepackage{fancyhdr,lastpage}
\pagestyle{fancy}
%\pagestyle{empty}      % Uncomment this to get rid of page numbers
\fancyhf{}\renewcommand{\headrulewidth}{0pt}
\fancyfootoffset{\marginparsep+\marginparwidth}
\newlength{\footpageshift}
\setlength{\footpageshift}
          {0.5\textwidth+0.5\marginparsep+0.5\marginparwidth-2in}
\lfoot{\hspace{\footpageshift}%
       \parbox{4in}{\, \hfill %
                    \arabic{page} of \protect\pageref*{LastPage} % +LP
%                    \arabic{page}                               % -LP
                    \hfill \,}}

% Finally, give us PDF bookmarks
\usepackage{color,hyperref}
\definecolor{darkblue}{rgb}{0.0,0.0,0.3}
\hypersetup{colorlinks,breaklinks,
            linkcolor=darkblue,urlcolor=darkblue,
            anchorcolor=darkblue,citecolor=darkblue}

%%%%%%%%%%%%%%%%%%%%%%%% End Document Setup %%%%%%%%%%%%%%%%%%%%%%%%%%%%


%%%%%%%%%%%%%%%%%%%%%%%%%%% Helper Commands %%%%%%%%%%%%%%%%%%%%%%%%%%%%

% The title (name) with a horizontal rule under it
% (optional argument typesets an object right-justified across from name
%  as well)
%
% Usage: \makeheading{name}
%        OR
%        \makeheading[right_object]{name}
%
% Place at top of document. It should be the first thing.
% If ``right_object'' is provided in the square-braced optional
% argument, it will be right justified on the same line as ``name'' at
% the top of the CV. For example:
%
%       \makeheading[\emph{Curriculum vitae}]{Your Name}
%
% will put an emphasized ``Curriculum vitae'' at the top of the document
% as a title. Likewise, a picture could be included:
%
%   \makeheading[\includegraphics[height=1.5in]{my_picutre}]{Your Name}
%
% the picture will be flush right across from the name.
\newcommand{\makeheading}[2][]%
        {\hspace*{-\marginparsep minus \marginparwidth}%
         \begin{minipage}[t]{\textwidth+\marginparwidth+\marginparsep}%
             {\large \bfseries #2 \hfill #1}\\[-0.15\baselineskip]%
                 \rule{\columnwidth}{1pt}%
         \end{minipage}}

% The section headings
%
% Usage: \section{section name}
%
% Follow this section IMMEDIATELY with the first line of the section
% text. Do not put whitespace in between. That is, do this:
%
%       \section{My Information}
%       Here is my information.
%
% and NOT this:
%
%       \section{My Information}
%
%       Here is my information.
%
% Otherwise the top of the section header will not line up with the top
% of the section. Of course, using a single comment character (%) on
% empty lines allows for the function of the first example with the
% readability of the second example.
\renewcommand{\section}[2]%
        {\pagebreak[3]\vspace{1.3\baselineskip}%
         \phantomsection\addcontentsline{toc}{section}{#1}%
         \hspace{0in}%
         \marginpar{
         \raggedright \scshape #1}#2}

% An itemize-style list with lots of space between items
\newenvironment{outerlist}[1][\enskip\textbullet]%
        {\begin{itemize}[#1]}{\end{itemize}%
         \vspace{-.6\baselineskip}}

% An environment IDENTICAL to outerlist that has better pre-list spacing
% when used as the first thing in a \section
\newenvironment{lonelist}[1][\enskip\textbullet]%
        {\vspace{-\baselineskip}\begin{list}{#1}{%
        \setlength{\partopsep}{0pt}%
        \setlength{\topsep}{0pt}}}
        {\end{list}\vspace{-.6\baselineskip}}

% An itemize-style list with little space between items
\newenvironment{innerlist}[1][\enskip\textbullet]%
        {\begin{compactitem}[#1]}{\end{compactitem}}

% An environment IDENTICAL to innerlist that has better pre-list spacing
% when used as the first thing in a \section
\newenvironment{loneinnerlist}[1][\enskip\textbullet]%
        {\vspace{-\baselineskip}\begin{compactitem}[#1]}
        {\end{compactitem}\vspace{-.6\baselineskip}}

% To add some paragraph space between lines.
% This also tells LaTeX to preferably break a page on one of these gaps
% if there is a needed pagebreak nearby.
\newcommand{\blankline}{\quad\pagebreak[3]}
\newcommand{\halfblankline}{\quad\vspace{-0.5\baselineskip}\pagebreak[3]}

% Uses hyperref to link DOI
\newcommand\doilink[1]{\href{http://dx.doi.org/#1}{#1}}
\newcommand\doi[1]{doi:\doilink{#1}}

% For \url{SOME_URL}, links SOME_URL to the url SOME_URL
\providecommand*\url[1]{\href{#1}{#1}}
% Same as above, but pretty-prints SOME_URL in teletype fixed-width font
\renewcommand*\url[1]{\href{#1}{\texttt{#1}}}

% For \email{ADDRESS}, links ADDRESS to the url mailto:ADDRESS
\providecommand*\email[1]{\href{mailto:#1}{#1}}
% Same as above, but pretty-prints ADDRESS in teletype fixed-width font
%\renewcommand*\email[1]{\href{mailto:#1}{\texttt{#1}}}

%\providecommand\BibTeX{{\rm B\kern-.05em{\sc i\kern-.025em b}\kern-.08em
%    T\kern-.1667em\lower.7ex\hbox{E}\kern-.125emX}}
%\providecommand\BibTeX{{\rm B\kern-.05em{\sc i\kern-.025em b}\kern-.08em
%    \TeX}}
\providecommand\BibTeX{{B\kern-.05em{\sc i\kern-.025em b}\kern-.08em
    \TeX}}
\providecommand\Matlab{\textsc{Matlab}}

%%%%%%%%%%%%%%%%%%%%%%%% End Helper Commands %%%%%%%%%%%%%%%%%%%%%%%%%%%

%%%%%%%%%%%%%%%%%%%%%%%%% Begin CV Document %%%%%%%%%%%%%%%%%%%%%%%%%%%%

\begin{document}
\makeheading{David Trail}

\section{Contact Information}
%
% NOTE: Mind where the & separators and \\ breaks are in the following
%       table.
%
% ALSO: \rcollength is the width of the right column of the table
%       (adjust it to your liking; default is 1.85in).
%
\newlength{\rcollength}\setlength{\rcollength}{2.5in}%
%
\begin{tabular}[t]{@{}p{\textwidth-\rcollength}p{\rcollength}}
11/6 Watson Crescent                   & \textit{Mobile:} +44 7816 930 167 \\
Edinburgh		& \textit{Email:} \email{mr.dj.trail@gmail.com} \\
EH11 1HB				& \textit{DoB:} 8th of January 1990 \\

\end{tabular}

\halfblankline

\section{About me}
What I'm looking for more than anything is to get into a Senior DevOps role, that has been my number one priority for the past two years and what I see myself doing for the foreseeable future. I have already made great strides towards that end in my own time and by selecting work for clients involved in high level Linux sysadmin and DevOps related technologies. \\
I know that my current Linux sysadmin abilities are extremely good as I have been running dedicated and cloud servers and VPSs since 2006, I have run them since before I started at university and continue to administer them to this day. \\
Though freelancing I've taught myself a great many skills in the world of sysadmin/DevOps but what I'm really looking for now is to hone those skills to perfection in a permanent work environment.

\halfblankline

\section{Work history}
% Back to Edinburgh!
\vspace{-8mm}
\begin{outerlist}
	\item[] \textbf{Freelance Linux System Administrator} \\
	\textit{September 2014 until today}
	\begin{innerlist}
		\item Repeat contracts for numerous clients generally in the area of Linux DevOps/sysadmin/software engineering
		\item Technologies include: Docker, Qemu/KVM, OpenVPN, Puppet, Python, Perl, Bash, AWS (Cloud Formation, EC2, S3, \ldots), CloudFlare, Git, MySQL, PostgreSQL, Nginx/Apache/uwsgi, etc
		\item Duties include: migration, configuration and set-up, security, service optimisation, monitoring, benchmarking, auto-scaling, networking, \ldots
	\end{innerlist}
	\item[] Client: Leo Hämäläinen
	\begin{innerlist}
		\item Leo has been my client for the past $\approx$ 7 months
		\item I have been contracted as a Linux sysadmin for their operations at \href{http://devoca.fi}{devoca.fi}, a backend for their hand-held label device and other products with hundreds of clients throughout Finland and elsewhere who rely on Devoca's servers to operate
		\item Duties have included: Upscaling operations; troubleshooting downtime; eradicating a rootkit that they'd contracted under a previous contractor; optimising both Apache and MySQL's configurations; planning for minimal downtime during migration periods; master-slave MySQL mirroring; backup script writing
		\item Technologies include: Apache Servermix, Apache HTTP, Exim mail server (MTA), Ubuntu 14.04, iptables firewall, fail2ban, MySQL, DKIM + SPF, DNS
		\item Communication has been key to my success in this role
	\end{innerlist}
	\item[] Client: CloudWorks
	\begin{innerlist}
		\item I was contracted to set up a continuous deployment strategy utilising AWS's Cloud Formation, EC2 and RDS alongside a Puppet master server to manage the deployed servers. This required writing from scratch a Cloud Formation script with a variable number of EC2 instances which would image themselves from an AMI which would connect to a Puppet master server and auto-configure themselves with an app hosted on Github.
		\item This has been the most challenging but also most enjoyable role I have done so far. Working within Linux, AWS and DevOps is where I feel completely at home and find myself loving the work that I do, impressing myself with my own abilities.
		\item Technologies: AWS, EC2, RDS, Route53, Puppet, PostgreSQL, troposphere, cloud-config, Cloud Formation
	\end{innerlist}
\end{outerlist}

\halfblankline

% Time spent in Salamanca
\vspace{-2mm}
\begin{outerlist}
	\item[] \textbf{Freelance Software Engineer} \\
	\textit{July 2013 to September 2014} (1 year 2 months)
	\begin{innerlist}
		\item Several large projects using Django + Python
		\item Web development
		\item Postgres + MySQL
		\item Bitcoin trading, investment and development
		\item Linux system administration
	\end{innerlist}
	\item[] Client: Éric Bisceglia
	\begin{innerlist}
		\item Worked with Éric and another software engineer for $\approx$ 6 months
		\item Containerising app instances to provide security in case of a security breach, ensuring uptime and redundancy for the web frontend/PostgreSQL backend on Microsoft's Azure platform
	\end{innerlist}
\end{outerlist}

\halfblankline

% Scotweb section
\vspace{-2mm}
\begin{outerlist}
\item[] \textbf{Software Engineer, Scotweb} \\
        \textit{June 2011 to July 2013} (2 years 1 month)
	\begin{innerlist}
		\item Commercial web development and maintenance
		\item Linux system administration
		\item Twisted Python / Python development
		\item Full stack deployment and configuration
		\item Web development (HTML/CSS/jQuery)
	\end{innerlist}
\end{outerlist}

% Ancient work section
\vspace{-2mm}
\begin{outerlist}
	\item[] \textbf{Village Orderly, Aberdeenshire council} \\
	\textit{July 2007 to September 2010} (3 years, 2 months)
\end{outerlist}

\vspace{-2mm}
\begin{outerlist}
	\item[] \textbf{Kitchen Porter, Valentinos restaurant} \\
	\textit{June 2004 to June 2006} (2 years)
\end{outerlist}

\newpage

\section{Education}
Computer Science BSc, \href{http://www.macs.hw.ac.uk/}{\textbf{Heriot-Watt University}} \\
\emph{September 2007 to May 2011} \\

\begin{tabular}{ c | c | c }
	2\textsuperscript{nd} year & 3\textsuperscript{rd} year & 4\textsuperscript{th} year \\
	\hline
	Database Management & Artificial Intelligence & Distributed Systems \\
	Discreet Mathematics & Operating Systems & Parallel Technologies \\
	Internet Communications & Computer Graphics & Data Mining \\
	Software Design & Professional Development & Machine Learning \\
	Formal Specifications & Concurrency & Biological computation \\
\end{tabular} \\


\section{Other}
\vspace{-5mm}
\begin{innerlist}
	\item Achieved Intermediate 1 Spanish (\href{http://eoisalamanca.centros.educa.jcyl.es/sitio/}{Escuela de Idiomas})
	\item Raised \pounds{1,050} for MND Scotland in a charity Skydive
	\item Maintain Github and Bitbucket profiles
	\item Always up to date with the latest technologies in DevOps and Software Engineering
\end{innerlist} 


\section{References}
\begin{tabular}{ r | c | c }
	Name & \textbf{Dr.~Greg Michaelson} & {\textbf{Mr.~Steve Tuttle}} \\
	%\hline
	Relation & Project supervisor & Client (DevOps, AWS)  \\
	Email & ~\href{mailto:G.Michaelson@hw.ac.uk}{G.Michaelson@hw.ac.uk} & ~\href{mailto:swt@drinkme.com}{swt@drinkme.com} \\
	Phone & ~+44 131 451 3422 & ~+61 400 601 689 \\
\end{tabular}

%\footnotetext{CV generated on \today, with \LaTeX}

\end{document}
